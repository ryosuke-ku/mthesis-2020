% 2018.01.08 Modified
% 2015.07.10 Modified
%
% mthesis.tex
%
\documentclass[12pt]{jarticle} % Japanese
\usepackage{comment}
%\documentclass[12pt]{article} % English
% if there are problems in the above regarding fonts, use this
% \documentclass[UTF8]{ctexart}

\usepackage[utf8]{inputenc}
%\usepackage{utf}
\usepackage{naist-jmthesis} %Japanese
%\usepackage{naist-mthesis} %English

\usepackage{graphicx}

%
% Page style
%
\pagestyle{final}       % Camera-Ready
%\pagestyle{draft}      % Draft
%
%
\lang{Japanese} % Japanese
%\lang{English} % English
%
% Student Number
%
\studentnumber{1811098}
%
% 修士論文 か 課題研究 かの選択
%
\doctitle{\mastersthesis}       % 修士論文
%\doctitle{\mastersreport}      % 課題研究
%
% 取得予定の修士号は 修士(工学) か 修士(理学) か ?
%
\major{\engineering}    % 工学
%\major{\science}       % 理学
%
% 日本語題目 (in LaTeX)
%
\title{ソースコードの類似性に基づいたテストコード\\自動推薦ツール}
%
% 日本語題目 (in plain text)
%
%   注: (in LaTeX)と同じ場合は指定する必要なし。
%       この情報は修士論文/課題研究には現れませんが、管理のために必要です。
%
\ptitle{太陽と月を利用したpiの低速計算アルゴリズムに関する理論的研究}
%
% 英語題目 (in LaTeX)
%
\etitle{Automatic Test Suite Recommendation System based on Code Clone Detection}
%
% 英語題目 (in plain text)
%
%   注: (in LaTeX)と同じ場合は指定する必要なし。
%       この情報は修士論文/課題研究には現れませんが、管理のために必要です。
%
\eptitle{Theoretical Studies on Low-Speed Calculation Algorithms of pi \\
Utilizing the Sun and the Moon}
%
% 日本語氏名 (in LaTeX)
%   (姓と名の間に空白を入れて下さい)
%
\author{倉地 亮介}
%
% 日本語氏名 (in plain text)
%
%   注: (in LaTeX)と同じ場合は指定する必要なし。
%       この情報は修士論文/課題研究には現れませんが、管理のために必要です。
%
\pauthor{}
%
% 欧文氏名 (in LaTeX)
%   (first name, last name の順に記入し、先頭文字のみを大文字にする。)
%
\eauthor{Ryosuke Kurachi}
% 別の例: \eauthor{Kurt G\"{o}del}
%
%
% 欧文氏名 (in plain text)
%
%   注: (in LaTeX)と同じ場合は指定する必要なし。
%       この情報は修士論文/課題研究には現れませんが、管理のために必要です。
%
\epauthor{}
% 別の例: \peauthor{Kurt Goedel}
%
%
% 論文提出年月日
%
\syear{2020}
\smonth{1}
\sday{28}
%
% 専攻の選択
%
\department{\infproc}  % 情報処理学
%\department{\infsys}    % 情報システム学
%\department{\bioinf}   % 情報生命科学
%\department{\infsci}    % 情報科学
%
%
% 審査委員(日本語)
%   (姓と名、名と称号の間に空白を入れて下さい)
%
%5人以上の場合,5人目以降は\addcmembers を使って宣言する。
%最大で合わせて8人まで宣言可能。
%主指導教員、副指導教員を明記する。両指導教員以外は委員。
%学外審査委員は、大学名を明記する
%
% 4人の場合
\cmembers{飯田 元 教授}{(主指導教員)}
         {井上 美智子 教授}{(副指導教員)}
         {市川 昊平 准教授}{(副指導教員)}
         {崔 恩瀞 准教授}{(京都工芸繊維大学)}
%
% 3人の場合
%\cmembers{○○ ○○ 教授}{(主指導教員)}
%         {○○ ○○ 教授}{(副指導教員)}
%         {○○ ○○ 准教授}{(副指導教員)}
%         {}{}
%
% 2人の場合
%\cmembers{○○ ○○ 教授}{(主指導教員)}
%         {○○ ○○ 教授}{(副指導教員)}
%          {}{}
%          {}{}
%
% 5人目の宣言
%\addcmembers{55 55 准教授}{(□□大学)}
%            {}{}
%            {}{}
%            {}{}
%
% 5〜6人目の宣言
%\addcmembers{55 55 准教授}{(□□大学)}
%            {66 66 准教授}{(□□大学)}
%            {}{}
%            {}{}
%
% 5〜7人目の宣言
%\addcmembers{55 55 准教授}{(□□大学)}
%            {66 66 准教授}{(□□大学)}
%            {77 77 准教授}{(□□大学)}
%            {}{}
%
% 5〜8人目の宣言
%\addcmembers{55 55 准教授}{(□□大学)}
%            {66 66 准教授}{(□□大学)}
%            {77 77 准教授}{(□□大学)}
%            {88 88 准教授}{(□□大学)}
%
%
% 審査委員(英語)
%     (first name, last name の順に記入し、先頭文字のみを大文字にする。
%       first name と last name の間に空白、
%       last name と 称号の間にカンマと空白を入れて下さい。)
%
% 5人以上の場合,5人目以降は\eaddcmembers を使って宣言する
% Supervisor, Co-supervisor, and Member must be specified.
% 4人の場合
\ecmembers{Professor XXX XXX}{(Supervisor)}
          {Professor XXX XXX}{(Co-supervisor)}
          {Associate Professor XXX XXX}{(Co-supervisor)}
          {Associate Professor XXX XXX}{(YY University)}
%
% 3人の場合
%\ecmembers{Professor XXX XXX}{(Supervisor)}
%          {Professor XXX XXX}{(Co-supervisor)}
%          {Associate Professor XXX XXX}{(Co-supervisor)}
%          {}{}
%
% 2人の場合
% \ecmembers{Professor XXX XXX}{(Supervisor)}
%           {Professor XXX XXX}{(Co-supervisor)}
%           {}{}
%           {}{}
%
% 5人目の宣言
%\eaddcmembers{Professor 55 55}{(YY University)}
%            {}{}
%            {}{}
%            {}{}
%
% 5〜6人目の宣言
%\eaddcmembers{Professor 55 55}{(YY University)}
%             {Professor 66 66}{(YY University)}
%             {}{}
%             {}{}
%
% 5〜7人目の宣言
%\eaddcmembers{Professor 55 55}{(YY University)}
%             {Professor 66 66}{(YY University)}
%             {Professor 77 77}{(YY University)}
%             {}{}
%
% 5〜8人目の宣言
%\eaddcmembers{Professor 55 55}{(YY University)}
%             {Professor 66 66}{(YY University)}
%             {Professor 77 77}{(YY University)}
%             {Professor 88 88}{(YY University)}
%
%
%
% キーワード5〜6個 (in LaTeX)
%
\keywords{類似コード検出, 推薦システム, ソフトウェアテスト, 単体テスト}
%
% キーワード5〜6個 (in plain text)
%
%   注: (in LaTeX)と同じ場合は記入する必要なし。
%       この情報は修士論文/課題研究には現れませんが、管理のために必要です。
%
\pkeywords{類似コード検出, 推薦システム, ソフトウェアテスト, 単体テスト}
%
% 5 or 6 Keywords (in LaTeX)
%
\ekeywords{clone detection, recommendation system, software testing, unit test}
%
% 5 or 6 Keywords (in plain text)
%
%   注: (in LaTeX)と同じ場合は記入する必要なし。
%       この情報は修士論文/課題研究には現れませんが、管理のために必要です。
%
\epkeywords{pi, astronomy, mathematics, computer, algorithm}
%
% 内容梗概 (in LaTeX)
%
%   注: 行の先頭が\\で始まらないようにすること。
%
\abstract{
ソフトウェアの品質確保の要と言えるソフトウェアテストを支援することは重要である.これまでに,テスト作成コストを削減するために様々な自動生成技術が提案されてきた.しかし,自動生成されたテストコードはテスト対象コードの作成経緯や意図に基づいて生成されていないという性質から後のメンテナンス活動を困難にさせる課題がある.この課題の解決方法として,既存テストの再利用が有効であると考えられる.そこで,本研究ではOSSプロジェクト上に存在する既存の品質が高いテストコード推薦するツールSuiteRecを提案する.SuiteRecは,類似コード検索ツールを用いてクローンペア間でのテスト再利用を考える.開発者から入力コードに対して類似コードを検出し,その類似コードに対応するテストスイートを開発者に推薦する.さらに,テストコードの良くない実装を表す指標を示すテストスメルを開発者に提示し,より品質の高いテストスイートを推薦できるように推薦順位を並び替える.提案ツールの評価では,被験者によってSuiteRecの使用した場合とそうでない場合でテストコードの作成してもらい,テスト作成をどの程度支援できるかを定量的および定性的に評価した.その結果,SuiteRecを利用した場合,(1) 条件分岐が多いプログラムのテストコードを作成する際にコードカバレッジの向上に効果的であること,(2) 作成したテストコードはテストスメルの数が少なく品質が高いこと,(3) 開発者はテストの作成を容易だと認識し,自身で作成したテストコードに自信が持てることが分かった.
}
%
% 内容梗概 (in plain text)
%
%   注: (in LaTeX)と同じ場合は記入する必要なし。
%       この情報は修士論文/課題研究には現れませんが、管理のために必要です。
%       改行する箇所には空白行を入れる。
%       行の先頭が\\で始まらないようにすること。
%
\pabstract{
ソフトウェアの品質確保の要と言えるソフトウェアテストを支援することは重要である.これまでに,テスト作成コストを削減するために様々な自動生成技術が提案されてきた.しかし,自動生成されたテストコードはテスト対象コードの作成経緯や意図に基づいて生成されていないという性質から後のメンテナンス活動を困難にさせる課題がある.この課題の解決方法として,既存テストの再利用が有効であると考えられる.そこで,本研究ではOSSプロジェクト上に存在する既存の品質が高いテストコード推薦するツールSuiteRecを提案する.SuiteRecは,類似コード検索ツールを用いてクローンペア間でのテスト再利用を考える.開発者から入力コードに対して類似コードを検出し,その類似コードに対応するテストスイートを開発者に推薦する.さらに,テストコードの良くない実装を表す指標を示すテストスメルを開発者に提示し,より品質の高いテストスイートを推薦できるように推薦順位を並び替える.提案ツールの評価では,被験者によってSuiteRecの使用した場合とそうでない場合でテストコードの作成してもらい,テスト作成をどの程度支援できるかを定量的および定性的に評価した.その結果,SuiteRecを利用した場合,(1) 条件分岐が多いプログラムのテストコードを作成する際にコードカバレッジの向上に効果的であること,(2) 作成したテストコードはテストスメルの数が少なく品質が高いこと,(3) 開発者はテストの作成を容易だと認識し,自身で作成したテストコードに自信が持てることが分かった.
}
%
% Abstract (in LaTeX)
%
%  注:  行の先頭が\\で始まらないようにすること。
%
\eabstract{
Automatically generated tests tend to be less read-able and maintainable since they often do not consider thelatent objective of the target code. Reusing existing tests mighthelp address this problem. To this end, we present SuiteRec, asystem that recommends reusable test suites based on code clonedetection. Given a java method, SuiteRec searches for its codeclones from a code base collected from open-source projects,and then recommends test suites of the clones. It also providesthe ranking of the recommended test suites computed basedon the similarity between the input code and the cloned code.We evaluate SuiteRec with a human study of ten students. Theresults indicate that SuiteRec successfully recommends reusabletest suites.
}
%
% Abstract (in plain text)
%
%   注: (in LaTeX)と同じ場合は記入する必要なし。
%       この情報は修士論文/課題研究には現れませんが、管理のために必要です。
%       改行する箇所には空白行を入れる。
%       行の先頭が\\で始まらないようにすること。
%
\epabstract{
Automatically generated tests tend to be less read-able and maintainable since they often do not consider thelatent objective of the target code. Reusing existing tests mighthelp address this problem. To this end, we present SuiteRec, asystem that recommends reusable test suites based on code clonedetection. Given a java method, SuiteRec searches for its codeclones from a code base collected from open-source projects,and then recommends test suites of the clones. It also providesthe ranking of the recommended test suites computed basedon the similarity between the input code and the cloned code.We evaluate SuiteRec with a human study of ten students. Theresults indicate that SuiteRec successfully recommends reusabletest suites.
}
%%%%%%%%%%%%%%%%%%%%%%%%% document starts here %%%%%%%%%%%%%%%%%%%%%%%%%%%%
\begin{document}
%
% 表紙 および アブストラクト
%
\titlepage
\cmemberspage
\firstabstract
\secondabstract
%
% 目次
%
\toc
\newpage
\listoffigures
%\newpage
\listoftables
%
% これ以降本文
%
\newpage
\section{はじめに}
\pagenumbering{arabic}
近年,ソフトウェアに求められる要件が高度化・多様化する一方,ユーザからはソフトウェアの品質確保やコスト削減に対する要求も増加している[1].その中でもソフトウェア開発全体のコストに占める割合が大きく,品質確保の要ともいえるソフトウェアテストを支援する技術への関心が高まっている\cite{b20}.しかし,現状ではテスト作成作業の大部分が人手で行われており,多くのテストを作成しようとするとそれに比例してコストも増加してしまう.このような背景から,ソフトウェアの品質を確保しつつコスト削減を達成するために,様々な自動化技術が提案されている\cite{b3},\cite{b16},\cite{b17},\cite{b18},\cite{b19}.

既存研究で提案されているEvoSuite\cite{b3}は,単体テスト自動生成における最先端のツールである.EvoSuiteは,対象コードを静的解析しプログラムを記号値で表現する.そして,対象コードの制御パスを通るような条件を集め,条件を満たす具体値を生成する.単体テストを自動生成することで,開発者は手作業での作成時間が自動生成によって節約することができ,またコードカバレッジを向上することができる.しかし,既存ツールによって自動生成されるテストコードは対象のコードの作成経緯や意図に基づいて生成されていないという性質から可読性が低く開発者に信用されていないことや後の保守作業を困難にするという課題がある\cite{b13},\cite{b14},\cite{b15}.このことは、自動生成ツールの実用的な利用の価値に疑問を提示させる.テストが失敗するたびに,開発者はテスト対象のプログラム内での不具合を原因を特定するまたは,テスト自体を更新する必要があるかどうかを判断する必要がある.自動生成されたテストは,自動生成によって得られる時間の節約よりも読みづらく,保守作業に助けになるというよりかむしろ邪魔するという結果が報告されている\cite{b1}.

我々は,この課題の解決するために既存テストの再利用が有効であると考える.本研究では,OSSに存在する既存の品質の高いテストコード推薦するツール{\sf SuiteRec}を提案する.推薦手法の基本となるアイディアは類似コード間でのテストコード再利用である.{\sf SuiteRec}は,入力コードに対して類似コードを検出し,その類似コードに対応するテストスイートを開発者に推薦する.さらに,テストコードの良くない実装を表す指標であるテストスメルを開発者に提示し,より品質の高いテストスイートを推薦できるように推薦順位がランキングされる.

提案ツールの評価では,被験者によって{\sf SuiteRec}の使用した場合とそうでない場合でテストコードの作成してもらい,テスト作成をどの程度支援できるかを定量的および定性的に評価した.その結果,{\sf SuiteRec}の利用は条件分岐が多く複雑なプログラムのテストコードを作成する際にコードカバレッジの向上に効果的であること,作成したテストコードの内のテストスメルの数が少なく品質が高いことが分かった.また,実験後のアンケートによる定性的な評価では,{\sf SuiteRec}を使用した場合被験者はテストコードの作成が容易になると認識し,また自分の作成したコードに自信が持てることが分かった.


%\ref{kako}節では、過去における研究について述べ、
%\ref{kadai}章では、現状と今後の課題について述べる。
%また、付録\ref{omake1}におまけその1を添付する。

\newpage
\section{背景}
\subsection{ソフトウェアテスト}
ソフトウェアテスト(以下,テスト)とは,ソフトウェア開発プロセスの中でも後半に実施される工程であり,品質確保おける最後の砦である.テストは,ソフトウェアが仕様書通りに動作することを確認すること,また不具合を検出し修正することでソフトウェアの品質を向上させることを目的として行われる.テストは図[1]で示すように,テスト計画,テスト設計,テスト実行,テスト管理とうい大きく4つのタスクで構成される.テスト計画タスクでは,開発全体の計画に基づき,テスト対象,スケジュール,各タスクの実施体制・リソース配分等の策定を行う.テスト設計タスクでは,設計書などソフトウェアの仕様が記述されたドキュメント等を基に,テストケースを作成する.テスト実行タスクでは,ソフトウェアを動作させ,それぞれのテストケースにおいてソフトウェアが期待通りの振る舞いをするかどうかを確認する.テスト管理タスクでは,テストの消化状況やソフトウェアの品質状況の確認を随時行い,テスト優先度やリソース見直しなどのアクションを行う.テスト工程のコスト削減のため,テスト実行タスクにおいて,単体テストではJUnit,結合テストSelenium,Appium等のテスト自動実行ツールの利用が進んでいる.しかし,テスト設計タスクはいまだ手動で行うことが多く,自動化技術の実用化および普及が期待されている.








\subsection{過去における研究}
\label{kako}


過去における研究としては\cite{alex_nips12}などがある。

過去における研究 過去における研究 過去における研究 
過去における研究 過去における研究 過去における研究 過去における研究 
過去における研究 過去における研究 過去における研究 過去における研究 

過去における研究 過去における研究 過去における研究 過去における研究 
過去における研究 過去における研究 過去における研究 過去における研究 
過去における研究 過去における研究 過去における研究 過去における研究 
過去における研究 過去における研究 過去における研究 過去における研究 
過去における研究 過去における研究 過去における研究 過去における研究 

過去における研究 過去における研究 過去における研究 過去における研究 
過去における研究 過去における研究 過去における研究 過去における研究 
過去における研究 過去における研究 過去における研究 過去における研究 
過去における研究 過去における研究 過去における研究 過去における研究 
過去における研究 過去における研究 過去における研究 過去における研究 

\subsection{研究の目的と意義}

研究の目的と意義 研究の目的と意義 研究の目的と意義 研究の目的と意義 
研究の目的と意義 研究の目的と意義 研究の目的と意義 研究の目的と意義 
研究の目的と意義 研究の目的と意義 研究の目的と意義 研究の目的と意義 
研究の目的と意義 研究の目的と意義 研究の目的と意義 研究の目的と意義 

研究の目的と意義 研究の目的と意義 研究の目的と意義 研究の目的と意義 
研究の目的と意義 研究の目的と意義 研究の目的と意義 研究の目的と意義 
研究の目的と意義 研究の目的と意義 研究の目的と意義 研究の目的と意義 
研究の目的と意義 研究の目的と意義 研究の目的と意義 研究の目的と意義 

研究の目的と意義 研究の目的と意義 研究の目的と意義 研究の目的と意義 
研究の目的と意義 研究の目的と意義 研究の目的と意義 研究の目的と意義 
研究の目的と意義 研究の目的と意義 研究の目的と意義 研究の目的と意義 
研究の目的と意義 研究の目的と意義 研究の目的と意義 研究の目的と意義 

\begin{figure}
\centerline{ここに図を書く}
\caption{これは図の例}
\end{figure}

\begin{table}
\centerline{ここに表を書く}
\caption{これは表の例}
\end{table}

研究の目的と意義 研究の目的と意義 研究の目的と意義 研究の目的と意義 
研究の目的と意義 研究の目的と意義 研究の目的と意義 研究の目的と意義 
研究の目的と意義 研究の目的と意義 研究の目的と意義 研究の目的と意義 
研究の目的と意義 研究の目的と意義 研究の目的と意義 研究の目的と意義 

研究の目的と意義 研究の目的と意義 研究の目的と意義 研究の目的と意義 
研究の目的と意義 研究の目的と意義 研究の目的と意義 研究の目的と意義 
研究の目的と意義 研究の目的と意義 研究の目的と意義 研究の目的と意義 
研究の目的と意義 研究の目的と意義 研究の目的と意義 研究の目的と意義 

研究の目的と意義 研究の目的と意義 研究の目的と意義 研究の目的と意義 
研究の目的と意義 研究の目的と意義 研究の目的と意義 研究の目的と意義 
研究の目的と意義 研究の目的と意義 研究の目的と意義 研究の目的と意義 
研究の目的と意義 研究の目的と意義 研究の目的と意義 研究の目的と意義 

研究の目的と意義 研究の目的と意義 研究の目的と意義 研究の目的と意義 
研究の目的と意義 研究の目的と意義 研究の目的と意義 研究の目的と意義 
研究の目的と意義 研究の目的と意義 研究の目的と意義 研究の目的と意義 
研究の目的と意義 研究の目的と意義 研究の目的と意義 研究の目的と意義 

研究の目的と意義 研究の目的と意義 研究の目的と意義 研究の目的と意義 
研究の目的と意義 研究の目的と意義 研究の目的と意義 研究の目的と意義 
研究の目的と意義 研究の目的と意義 研究の目的と意義 研究の目的と意義 
研究の目的と意義 研究の目的と意義 研究の目的と意義 研究の目的と意義 

研究の目的と意義 研究の目的と意義 研究の目的と意義 研究の目的と意義 
研究の目的と意義 研究の目的と意義 研究の目的と意義 研究の目的と意義 
研究の目的と意義 研究の目的と意義 研究の目的と意義 研究の目的と意義 
研究の目的と意義 研究の目的と意義 研究の目的と意義 研究の目的と意義 

研究の目的と意義 研究の目的と意義 研究の目的と意義 研究の目的と意義 
研究の目的と意義 研究の目的と意義 研究の目的と意義 研究の目的と意義 
研究の目的と意義 研究の目的と意義 研究の目的と意義 研究の目的と意義 
研究の目的と意義 研究の目的と意義 研究の目的と意義 研究の目的と意義 

研究の目的と意義 研究の目的と意義 研究の目的と意義 研究の目的と意義 
研究の目的と意義 研究の目的と意義 研究の目的と意義 研究の目的と意義 
研究の目的と意義 研究の目的と意義 研究の目的と意義 研究の目的と意義 
研究の目的と意義 研究の目的と意義 研究の目的と意義 研究の目的と意義 

研究の目的と意義 研究の目的と意義 研究の目的と意義 研究の目的と意義 
研究の目的と意義 研究の目的と意義 研究の目的と意義 研究の目的と意義 
研究の目的と意義 研究の目的と意義 研究の目的と意義 研究の目的と意義 
研究の目的と意義 研究の目的と意義 研究の目的と意義 研究の目的と意義 

研究の目的と意義研究の目的と意義研究の目的と意義研究の目的と意義 
研究の目的と意義研究の目的と意義研究の目的と意義研究の目的と意義 
研究の目的と意義研究の目的と意義研究の目的と意義研究の目的と意義 
研究の目的と意義研究の目的と意義研究の目的と意義研究の目的と意義 

研究の目的と意義研究の目的と意義研究の目的と意義研究の目的と意義 
研究の目的と意義研究の目的と意義研究の目的と意義研究の目的と意義 
研究の目的と意義研究の目的と意義研究の目的と意義研究の目的と意義 
研究の目的と意義研究の目的と意義研究の目的と意義研究の目的と意義 

研究の目的と意義研究の目的と意義研究の目的と意義研究の目的と意義 
研究の目的と意義研究の目的と意義研究の目的と意義研究の目的と意義 
研究の目的と意義研究の目的と意義研究の目的と意義研究の目的と意義 
研究の目的と意義研究の目的と意義研究の目的と意義研究の目的と意義 


\newpage

This page is written in English. This page is written in English. 
This page is written in English. This page is written in English. 
This page is written in English. This page is written in English. 
This page is written in English. This page is written in English. 

This page is written in English. This page is written in English. 
This page is written in English. This page is written in English. 
This page is written in English. This page is written in English. 
This page is written in English. This page is written in English. 

This page is written in English. This page is written in English. 
This page is written in English. This page is written in English. 
This page is written in English. This page is written in English. 
This page is written in English. This page is written in English. 

This page is written in English. This page is written in English. 
This page is written in English. This page is written in English. 
This page is written in English. This page is written in English. 
This page is written in English. This page is written in English. 

This page is written in English. This page is written in English. 
This page is written in English. This page is written in English. 
This page is written in English. This page is written in English. 
This page is written in English. This page is written in English. 

This page is written in English. This page is written in English. 
This page is written in English. This page is written in English. 
This page is written in English. This page is written in English. 
This page is written in English. This page is written in English. 

This page is written in English. This page is written in English. 
This page is written in English. This page is written in English. 
This page is written in English. This page is written in English. 
This page is written in English. This page is written in English. 
This page is written in English. This page is written in English. 
This page is written in English. This page is written in English. 
This page is written in English. This page is written in English. 
This page is written in English. This page is written in English. 

This page is written in English. This page is written in English. 
This page is written in English. This page is written in English. 
This page is written in English. This page is written in English. 
This page is written in English. This page is written in English. 
This page is written in English. This page is written in English. 
This page is written in English. This page is written in English. 
This page is written in English. This page is written in English. 
This page is written in English. This page is written in English. 

This page is written in English. This page is written in English. 
This page is written in English. This page is written in English. 
This page is written in English. This page is written in English. 
This page is written in English. This page is written in English. 
This page is written in English. This page is written in English. 
This page is written in English. This page is written in English. 
This page is written in English. This page is written in English. 
This page is written in English. This page is written in English. 


\newpage
\section{現状と今後の課題}
\label{kadai}

現状と今後の課題 現状と今後の課題 現状と今後の課題 現状と今後の課題 
現状と今後の課題 現状と今後の課題 現状と今後の課題 現状と今後の課題 
現状と今後の課題 現状と今後の課題 現状と今後の課題 現状と今後の課題 
現状と今後の課題 現状と今後の課題 現状と今後の課題 現状と今後の課題 

現状と今後の課題 現状と今後の課題 現状と今後の課題 現状と今後の課題 
現状と今後の課題 現状と今後の課題 現状と今後の課題 現状と今後の課題 
現状と今後の課題 現状と今後の課題 現状と今後の課題 現状と今後の課題 
現状と今後の課題 現状と今後の課題 現状と今後の課題 現状と今後の課題 

現状と今後の課題 現状と今後の課題 現状と今後の課題 現状と今後の課題 
現状と今後の課題 現状と今後の課題 現状と今後の課題 現状と今後の課題 
現状と今後の課題 現状と今後の課題 現状と今後の課題 現状と今後の課題 
現状と今後の課題 現状と今後の課題 現状と今後の課題 現状と今後の課題 

%
% 謝辞
%
\acknowledgements

Thank you. Thank you.
%
% 参考文献
% ここでは \reference を使って、自分でリストを作るか、BibTeX を使って
% リストをつくって下さい。この例では BibTeX を作るような形式になってい
% ます。
%
\newpage
% \reference
\bibliographystyle{plain}
\bibliography{mthesis}
%
% 付録
%
\appendix

\section{おまけその1}
\label{omake1}

これはおまけです。これはおまけです。これはおまけです。これはおまけです。
これはおまけです。これはおまけです。これはおまけです。これはおまけです。
これはおまけです。これはおまけです。これはおまけです。これはおまけです。
これはおまけです。これはおまけです。これはおまけです。これはおまけです。

\begin{figure}
\centerline{これはおまけの図です。}
\caption{おまけの図}
\end{figure}


\section{おまけその2}

これもおまけです。これもおまけです。これもおまけです。これもおまけです。
これもおまけです。これもおまけです。これもおまけです。これもおまけです。
これもおまけです。これもおまけです。これもおまけです。これもおまけです。
これもおまけです。これもおまけです。これもおまけです。これもおまけです。

\end{document}

